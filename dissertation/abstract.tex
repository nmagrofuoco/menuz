\chapter*{Abstract}
In 1994, \textsc{Sears} and \textsc{Shneiderman} proposed the changing concept 
of split menu. They radically changed the way we design menus until nowadays. 
Unfortunately, their guidelines haven't evolved in 20 years and the advent of 
smartphones have led HCI researchers to new and different usability issues. 
Based on their initial study, this master thesis aims to conceive, design and 
test a split menu adapted to smartphone resolutions.\newline

Along the way, approaches from various researchers have influenced our 
experiment and diverse menu organizations were also designed and tested. An 
experimental method was conducted combining traditional, split, responsive, 
minimised and mixed-initiative menus. The objective of this method was 
to assess the usability of these new menu organizations on smartphones. 
Usability was studied along with 3 interesting properties: (1) 
effectiveness, (2) efficiency and (3) user satisfaction.\newline

The experiment proved that split menu may not be the ideal solution for 
smartphones. Another novative menu organization called \enquote{responsive} 
showed a better usability. This dissertation aims to explain the development of 
the experiment and argue the analysis of its results.