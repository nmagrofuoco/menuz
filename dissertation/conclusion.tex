\chapter{Conclusion}
Conceiving, designing and testing new menu organizations was a long and tough 
task. First, it was necessary to understand and summarize previous researches. 
It was an essential step to extract the currently available knowledge about 
menu usability. Then, it was essential to formulate an initial set of 
hypotheses in order to guide the development of the experimental method. 
Based on these assumptions and this experimental method, an Android application 
has been
implemented to conduct the experiment and results were extracted from it. After 
a torough analysis of these results, we were finally able to confirm, reverse 
and update our hypotheses.\newline

The experimental method set up during the study helped us to confirm many 
assumptions. First, we observed that all new menu organizations were beneficial 
in reducing the average error rate, except for the split menu that provided 
the same error rate as the traditional menu. Then, we proved that the 
responsive and the 4 mixed-initiative menus helped users to reduce their average 
selection time and enhance their productivity. Moreover, users showed a higher 
preference rate for the responsive and the traditional menus. Finally, we 
have shown that guidance informations can help users to understand the overal 
operation of a menu organization.\newline

In conclusion, the split menu organization developed by \textsc{Sears} and 
\textsc{Shneiderman} is not the ideal solution for smartphone resolutions. The 
traditional and responsive menus received the highest user preference. The 
responsive menu also proved to be the most performant one in terms of 
usability. Therefore, the responsive menu designed by Yusuke \textsc{Fukazawa} 
with a specific focus towards small touch screens appears to be the most 
promising menu organization for smartphones. Further experiments should focus 
on this novative menu organization.

\section{Further}
Some matters remain unchallenged at the end of this study. Further 
improvements and additional issues remain to be accomplished, improved and 
set to keep enhancing our knowledge base about HCI and menu usability. This 
section aims to enumerate some of these matters.

\begin{itemize}
 \item \textbf{Large menu}: according to Khalid \textsc{Al-Omar} and Dimitrios 
\textsc{Rigas}, our set of menu items should be considered as a \enquote{small 
menu}. They tested new menu organizations with both a small set of 17 items and 
a larger set of 29 items. During their controlled experiment, they identified 
varying preferences between these menus. Therefore, the responsive menu may not 
be the ideal solution with larger menus.
 \item \textbf{Hot list}: according to \textsc{Sears} and 
\textsc{Shneiderman}, the hot list of a split menu should not exceed 4 items. 
During the experiment, we decided to implement a hot list made of 3 items. A 
split menu organization displaying a different hot list length may eventually 
be preferred by users and/or enhance their productivity.
 \item \textbf{Learning effect}: some subjects of the experiment were concerned 
about the learning effect between each session. Indeed, a few ones recognized 
to know the items better after a few evaluation sessions. Therefore, the 
final results may include some sort of bias.
 \item \textbf{Evaluation session}: some menu organizations are now 
acknowledged to be more usable than other ones with smartphones. We should 
conduct a second experiment which targets specifically these menu organizations 
and offers longer evaluation sessions. Indeed, a set of 10 selections for 
each menu seems like a short session in comparison to previous studies.
 \item \textbf{Keystroke menu}: during our research, we implemented another 
type of menu organization called \enquote{keystroke}. It consists of a 
traditional menu overhung with a search bar. This feature allows users 
to type the name of the desired item and watch the menu updated in real time. 
Unfortunately, the conducted experiment was already quite long and we decided 
to set aside this menu organization.
 \item \textbf{Log file}: a final improvement to the Android application would 
be to store each user's actions in a log file for a deeper analysis of their 
behaviours with new menu organizations.
\end{itemize}
