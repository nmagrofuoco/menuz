\cleardoublepage
\chapter{State of the Art}

The state of the art is the starting point of a study. A critical 
review of the research question is always a necessary prerequisite. It allows 
to identify the findings, nuances, authors and remaining issues of an area of 
study. This literature survey is illustrated as a time line (see Figure 
\ref{fig:timeline}) which depicts the evolution of knowledge on the effects 
of menu organization on user experience. It starts with the premises, the 
initial researches that began almost 40 years ago. Then it 
describes the advent of split menu - an important change in menu organization - 
and tries to clarify the debate between adaptive and adaptable approaches. It 
also describes an interesting work about responsive menus. Finally it brings a 
final note about the learning outcomes of these previous researches.

\begin{figure}[ht!]
      \centering
    \begin{tikzpicture}[node distance=0.8cm]
	\tikzstyle{circ_xs}
	  =[draw, circle, text width=0.5pt, inner sep=0.5pt]
	\tikzstyle{rect_lg}
	  =[draw, rectangle, text width=3cm, minimum height=.5cm, inner sep=0pt]
	\tikzstyle{rect_xlg}
	  =[draw, rectangle, text width=4.9cm, minimum height=.5cm, inner 
sep=0pt]
	
	\node[draw=none] (start)
	  {};
	  
	\node[circ_xs,draw=black,fill=black] (80)
	  [right=0.5cm of start] {};
	\node[] (1980)
	  [below=0.5cm of 80] {1980};
	  
	\node[rect_lg,fill=red!60,draw=none] (period_premises)
	  [above right=0pt of 80] {};
	\node[outer sep=0pt, inner sep=0pt] (empty1)
	  [above=1pt of period_premises] {};
	\node[] (premises)
	  [above=0.5cm of empty1] {Premises};
	  
	\node[circ_xs,draw=black,fill=black] (90)
	  [right=3cm of 80] {};
	\node[] (1990)
	  [below=0.5cm of 90] {1990};
	  
	\node[circ_xs,draw=black,fill=black] (94)
	  [right=1.2cm of 90] {};
	\node[] (1994)
	  [below=0.5cm of 94] {1994};
	
	\node[rect_xlg,fill=blue!60,draw=none] (period_split)
	  [above right=0pt of 94] {};
	\node[outer sep=0pt, inner sep=0pt] (empty2)
	  [above=1pt of period_split] {};
	\node[] (split_expe)
	  [above=2.2cm of empty2] {The advent of split menu};
	  
	\node[circ_xs,draw=black,fill=black] (00)
	  [right=1.8cm of 94] {};
	\node[] (2000)
	  [below=0.5cm of 00] {2000};
	  
	\node[rect_lg,fill=green!60,draw=none] (period_adapt)
	  [below right=0pt of 00] {};
	\node[outer sep=0pt, inner sep=0pt] (empty3)
	  [below=1pt of period_adapt] {};
	\node[] (adapt)
	  [below=1.5cm of empty3] {Adaptive vs. Adaptable approach};
	
	\node[circ_xs,draw=black,fill=black] (09)
	  [right=2.7cm of 00] {};
	\node[] (2009)
	  [above=1.6cm of 09] {2009};
	\node[] (mobile)
	  [below=0cm of 2009] {Responsive menus};
	
	\node[circ_xs,draw=black,fill=black] (10)
	  [right=0.22cm of 09] {};
	\node[] (2010)
	  [below=0.5cm of 10] {2010};
	
	\node[circ_xs,draw=black,fill=black] (17)
	  [right=2.1cm of 10] {};
	\node[] (2017)
	  [below=0.5cm of 17] {2017};
	  
	\node[circ_xs,draw=white,fill=white] (20)
	  [right=0.4cm of 17] {};
	  
	\node[circ_xs,draw=white,fill=white] (end)
	  [right=0.3cm of 20] {};
	  
	\draw[-] (start)--(80);
	\draw[-] (80)--(90);
	\draw[-] (90)--(94);
	\draw[-] (94)--(00);
	\draw[-] (00)--(09);
	\draw[-] (09)--(10);
	\draw[-] (10)--(17);
	\draw[-] (17)--(20);
	\draw[-latex,line width=0.1cm] (20)--(end);
	
	\draw[-] (80)--(1980);
	\draw[-] (90)--(1990);
	\draw[-] (94)--(1994);
	\draw[-] (00)--(2000);
	\draw[-] (09)--(mobile);
	\draw[-] (10)--(2010);
	\draw[-] (17)--(2017);
	
	\draw[-] (premises)--(empty1);
	\draw[-] (split_expe)--(empty2);
	\draw[-] (adapt)--(empty3);
    \end{tikzpicture}
    \caption{Time line of major research topics from 1980 to 2017.}
  \label{fig:timeline}
\end{figure}

\section{Premises}

Back in the early 80’s, a few researchers conducted a first set of experiments 
to better understand the effects of menu organization on user experience and 
user performance. However the scope of analysis was restrained to static and 
dynamic menu organizations only. Most of these studies resulted to be partially 
useful but they set interesting premises for the next ones.\newline
 
According to \textsc{Sears} and \textsc{Shneiderman} \cite{sears}, 
\textsc{Card} \cite{card} was the first researcher to show interest in assessing 
menu usability. In 1982 already, \textsc{Card} conducted an experiment based on 
3 menu organizations: (1) alphabetically ordered menu, (2) categorically ordered 
menu and (3) randomly ordered menu. These menus are depicted by Figure 
\ref{fig:card_menus}. \textsc{Card} proved that the alphabetically ordered 
menu was the fastest whereas randomly ordered menu was the slowest ones. He 
also assumed that a \textit{meaningful organization} - such as 
alphabetical or categorical ordering - could be beneficial for user 
experience.\newline

\begin{figure}[!ht]
    \centering
    \begin{tikzpicture}[node distance=0.8cm]
        \tikzstyle{item}
	  =[draw, fill=gray!20, rectangle, text width=2cm, 
	  minimum height=0.5cm, text centered]
	
	\node[] (1)
	  {(1)};
	\node[item] (A)
	  [below=3pt of 1]{Item A};
	\node[item] (B)
	  [below=3pt of A] {Item B};
	\node[item] (C)
	  [below=3pt of B] {Item C};
	\node[item] (D)
	  [below=3pt of C] {Item D};
	\node[item] (E)
	  [below=3pt of D] {Item E};
	\node[item] (F)
	  [below=3pt of E] {Item F};
	\node[item] (G)
	  [below=3pt of F] {Item G};
	\node[item] (H)
	  [below=3pt of G] {Item H};
	\node[item] (I)
	  [below=3pt of H] {Item I};
	\node[item] (J)
	  [below=3pt of I] {Item J};
	  
	\node[] (2)
	  [right=3cm of 1] {(2)};
	\node[item] (cat11)
	  [below=3pt of 2] {Item 1.1};
	\node[item] (cat12)
	  [below=3pt of cat11] {Item 1.2};
	\node[item] (cat21)
	  [below=3pt of cat12] {Item 2.1};
	\node[item] (cat31)
	  [below=3pt of cat21] {Item 3.1};
	\node[item] (cat32)
	  [below=3pt of cat31] {Item 3.2};
	\node[item] (cat33)
	  [below=3pt of cat32] {Item 3.3};
	\node[item] (cat41)
	  [below=3pt of cat33] {Item 4.1};
	\node[item] (cat42)
	  [below=3pt of cat41] {Item 4.2};
	\node[item] (cat51)
	  [below=3pt of cat42] {Item 5.1};
	\node[item] (cat61)
	  [below=3pt of cat51] {Item 6.1};
	  
	\node[] (3)
	  [right=3cm of 2]{(3)};
	\node[item] (F2)
	  [below=3pt of 3] {Item F};
	\node[item] (D2)
	  [below=3pt of F2] {Item D};
	\node[item] (A2)
	  [below=3pt of D2] {Item A};
	\node[item] (C2)
	  [below=3pt of A2] {Item C};
	\node[item] (H2)
	  [below=3pt of C2] {Item H};
	\node[item] (G2)
	  [below=3pt of H2] {Item G};
	\node[item] (I2)
	  [below=3pt of G2] {Item I};
	\node[item] (B2)
	  [below=3pt of I2] {Item B};
	\node[item] (E2)
	  [below=3pt of B2] {Item E};
	\node[item] (J2)
	  [below=3pt of E2] {Item J};
    \end{tikzpicture}
    \caption{\textsc{Card}'s menu organizations with 10 items : (1) 
alphabetically, (2) 
categorically and (3) randomly ordered menus.}
    \label{fig:card_menus}
\end{figure}

In 1987, \textsc{Somberg} \cite{somberg} also investigated the effects of menu 
organization on user performance. He replaced \textsc{Card}’s categorically 
ordered menu 
with a new approach based on probability of selection. A probability was bound 
to each item and was modified at each step of the test, displaying different 
menu organizations throughout the entire study. The alphabetical menu was also 
subject to changes as the first displayed letter was chosen randomly from one 
iteration to another. The randomly, alphabetically and probability ordered 
menus were then considered as dynamic menus. These designs are depicted by 
Figure \ref{fig:somberg_menus} which highlights two potential iterations for 
each dynamic menu. The 4th menu organization proposed by the author was called 
\enquote{positionally constant} and consisted of a static alphabetically 
ordered menu. \textsc{Somberg} proved that keeping the items in \textit{fixed 
locations} was more 
performant than allowing the items to move within the menu. Indeed the 
positionally constant menu was prefered and resulted in better user performance 
than dynamic ones.\newline

\begin{figure}[!ht]
    \centering
    \begin{tikzpicture}[node distance=0.8cm]
        \tikzstyle{item}
	  =[draw, fill=gray!20, rectangle, text width=2cm, 
	  minimum height=0.5cm, text centered]
	
	\node[] (1a)
	  {(1a)};
	\node[item] (A)
	  [below=3pt of 1a]{Item A};
	\node[item] (B)
	  [below=3pt of A] {Item B};
	\node[item] (C)
	  [below=3pt of B] {Item C};
	\node[item] (D)
	  [below=3pt of C] {Item D};
	\node[item] (E)
	  [below=3pt of D] {Item E};
	\node[item] (F)
	  [below=3pt of E] {Item F};
	\node[item] (G)
	  [below=3pt of F] {Item G};
	\node[item] (H)
	  [below=3pt of G] {Item H};
	\node[item] (I)
	  [below=3pt of H] {Item I};
	\node[item] (J)
	  [below=3pt of I] {Item J};
	 
	\node[] (1b)
	  [right=2cm of 1a] {(1b)};
	\node[item] (D3)
	  [below=3pt of 1b] {Item D};
	\node[item] (E3)
	  [below=3pt of D3] {Item E};
	\node[item] (F3)
	  [below=3pt of E3] {Item F};
	\node[item] (G3)
	  [below=3pt of F3] {Item G};
	\node[item] (H3)
	  [below=3pt of G3] {Item H};
	\node[item] (I3)
	  [below=3pt of H3] {Item I};
	\node[item] (J3)
	  [below=3pt of I3] {Item J};
	\node[item] (A3)
	  [below=3pt of J3]{Item A};
	\node[item] (B3)
	  [below=3pt of A3] {Item B};
	\node[item] (C3)
	  [below=3pt of B3] {Item C};

	\node[] (2a)
	  [right=3cm of 1b]{(2a)};
	\node[item] (F2)
	  [below=3pt of 2a] {Item F};
	\node[item] (D2)
	  [below=3pt of F2] {Item D};
	\node[item] (A2)
	  [below=3pt of D2] {Item A};
	\node[item] (C2)
	  [below=3pt of A2] {Item C};
	\node[item] (H2)
	  [below=3pt of C2] {Item H};
	\node[item] (G2)
	  [below=3pt of H2] {Item G};
	\node[item] (I2)
	  [below=3pt of G2] {Item I};
	\node[item] (B2)
	  [below=3pt of I2] {Item B};
	\node[item] (E2)
	  [below=3pt of B2] {Item E};
	\node[item] (J2)
	  [below=3pt of E2] {Item J};
	  
	\node[] (2b)
	  [right=2cm of 2a]{(2b)};
	\node[item] (F4)
	  [below=3pt of 2b] {Item J};
	\node[item] (D4)
	  [below=3pt of F4] {Item D};
	\node[item] (A4)
	  [below=3pt of D4] {Item B};
	\node[item] (C4)
	  [below=3pt of A4] {Item G};
	\node[item] (H4)
	  [below=3pt of C4] {Item A};
	\node[item] (G4)
	  [below=3pt of H4] {Item I};
	\node[item] (I4)
	  [below=3pt of G4] {Item E};
	\node[item] (B4)
	  [below=3pt of I4] {Item H};
	\node[item] (E4)
	  [below=3pt of B4] {Item C};
	\node[item] (J4)
	  [below=3pt of E4] {Item F};
	 
	\node[] (3a)
	  [below right=1cm and 2cm of J] {(3a)};
	\node[item] (F5)
	  [below=3pt of 3a] {Item G};
	\node[item] (D5)
	  [below=3pt of F5] {Item C};
	\node[item] (A5)
	  [below=3pt of D5] {Item E};
	\node[item] (C5)
	  [below=3pt of A5] {Item A};
	\node[item] (H5)
	  [below=3pt of C5] {Item I};
	\node[item] (G5)
	  [below=3pt of H5] {Item J};
	\node[item] (I5)
	  [below=3pt of G5] {Item H};
	\node[item] (B5)
	  [below=3pt of I5] {Item F};
	\node[item] (E5)
	  [below=3pt of B5] {Item B};
	\node[item] (J5)
	  [below=3pt of E5] {Item D};
	  
	\node[] (MostProb3a)
	  [left=0.5cm of F5] {Most probable in 3a};
	\node[] (LessProb3a)
	  [left=0.5cm of J5] {Less probable in 3a};
	  
	\node[] (3b)
	  [right=2cm of 3a]{(3b)};
	\node[item] (F6)
	  [below=3pt of 3b] {Item C};
	\node[item] (D6)
	  [below=3pt of F6] {Item E};
	\node[item] (A6)
	  [below=3pt of D6] {Item G};
	\node[item] (C6)
	  [below=3pt of A6] {Item A};
	\node[item] (H6)
	  [below=3pt of C6] {Item J};
	\node[item] (G6)
	  [below=3pt of H6] {Item I};
	\node[item] (I6)
	  [below=3pt of G6] {Item B};
	\node[item] (B6)
	  [below=3pt of I6] {Item H};
	\node[item] (E6)
	  [below=3pt of B6] {Item F};
	\node[item] (J6)
	  [below=3pt of E6] {Item D}; 
	
	\node[] (MostProb3b)
	  [right=0.5cm of F6] {Most probable in 3b};
	\node[] (LessProb3b)
	  [right=0.5cm of J6] {Less probable in 3b};
	
	\draw[-latex] (1a)--(1b);
	\draw[-latex] (2a)--(2b);
	\draw[-latex] (3a)--(3b);
	\draw[-latex,dashed] (MostProb3a)--(LessProb3a);
	\draw[-latex,dashed] (MostProb3b)--(LessProb3b);
    \end{tikzpicture}
    \caption{Illustrating 2 potential iterations (a) and (b) for each 
\textsc{Somberg}'s dynamic menu organization with 10 items : (1) 
alphabetically, (2) 
randomly and (3) probability ordered menus. (1a) can be considered as the 
static alphabetically ordered menu described by Somberg.}
    \label{fig:somberg_menus}
\end{figure}

In 1985, \textsc{Greenberg} and \textsc{Witten} \cite{greenberg_witten} 
investigated the benefits 
of organizing a menu based on a-priori set of frequencies and updating these 
frequencies according to user’s selections. \textsc{Mitchell} and 
\textsc{Shneiderman} 
\cite{mitchell_shneiderman} provided the first convincing results about such a 
frequency reordering in 1989. They conducted an experiment in which they 
compared a static menu to an automatically reorganized menu based on frequency. 
At first exposure, users preferred, were faster and made fewer errors with the 
static menu. After practice, \textsc{Mitchell} and \textsc{Shneiderman} proved 
that user 
performance was not a problem anymore with the dynamic menu. However this menu 
organization was still not preferred over the static one. They ended up by 
concluding that automatically updating menu organization to reflect the current 
usage patterns might be useful but could also lead to several problems.

\section{The Advent of Split Menu}

In 1994, \textsc{Sears} and \textsc{Shneiderman} \cite{sears} proposed the 
changing concept of 
\textit{split menu}. Based on the early work of \textsc{Greenberg} and 
\textsc{Witten}, \textsc{Smith} and 
\textsc{Mosier} \cite{smith_mosier} were already suggesting in 1986 
\enquote{where some 
data items are used more frequently than others, consider grouping those items 
at the top of the display}. Their solution was articulated in 3 steps: (1) 
maintaining the standard selection mechanism, (2) eliminating the need for 
users to remember additional commands and (3) reorganizing menu items based on 
usage frequency such that the most frequently selected items are moved to the 
top of the menu. 
The founding notions of split menu were finally set.\newline

\textsc{Sears} and \textsc{Shneiderman} defined a split menu as a menu created 
by splitting a menu 
into two sections such as frequently selected items are placed in the top 
section and infrequently selected items are placed in the bottom section. Such 
a 
menu organization proves useful \textit{if and only if} a small subset of the 
menu items represent the majority of selections. This menu organization is 
illustrated by Figure \ref{fig:sears_split} with a dark gray top section above 
the usual alphabetically ordered menu. Notice that most frequently 
selected items are moved to the top section and do not appear anymore in the 
remaining alphabetically ordered menu. Both authors defined a set of 
preliminary guidelines based on initial observations about user's selections 
and refined them into a cognitive model. These guidelines are still applicable 
nowadays and will be explained later in the report considering their 
importance. 
The cognitive model allowed both researchers to prove that (1) a linear model 
could be applied to the infrequently selected items and (2) a logarithmic model 
called \textit{Fitts’ Law} could be applied to the most frequently selected 
items. The linear model illustrates the fact that users spend most of their 
time 
searching an unfamiliar item in a linear fashion by scrolling down the 
given menu. The logarithmic approach proves that users do not scan the entire 
menu when searching for a familiar item but use the order of the menu instead. 
For example, a subject will start at the middle of the menu and will then 
decide 
if the desired item should be above or below this position.\newline

\begin{figure}[!ht]
    \centering
    \begin{tikzpicture}[node distance=0.8cm]
        \tikzstyle{item}
	  =[draw, fill=gray!20, rectangle, text width=2cm, 
	  minimum height=0.5cm, text centered]
	
	\node[item,fill=gray!50] (A)
	  {Item E};
	\node[item,fill=gray!50] (B)
	  [below=3pt of A] {Item G};
	\node[item,fill=gray!50] (C)
	  [below=3pt of B] {Item I};
	\node[item,fill=gray!50] (D)
	  [below=3pt of C] {Item J};
	\node[item] (E)
	  [below=3pt of D] {Item A};
	\node[item] (F)
	  [below=3pt of E] {Item B};
	\node[item] (G)
	  [below=3pt of F] {Item C};
	\node[item] (H)
	  [below=3pt of G] {Item D};
	\node[item] (J)
	  [below=3pt of H] {Item F};
	\node[item] (H1)
	  [below=3pt of J] {Item H};
    \end{tikzpicture}
    \caption{\textsc{Sears}' split menu organization with 10 items. The 4 most 
frequently selected items are moved to the top section and displayed in 
alphabetical order.}
    \label{fig:sears_split}
\end{figure}

Split menu provided the missing concept to conciliate both user performance and 
user preference. Indeed \textsc{Sears} and \textsc{Shneiderman} conducted two 
in situ usability 
studies and managed to prove interesting properties: (1) split menu allows to 
enhance user performance only after a short period of one week adjustment, (2) 
split menu allows 17-58\% time savings, (3) 9 out of 13 users expressed 
preference for split menu, (4) 3 out of 13 users expressed no preference and 
(4) split menu provides benefits of both frequency ordered and alphabetically 
ordered menus. In conclusion, \textsc{Sears} and \textsc{Shneiderman} proved 
that frequency reordering was not sufficient alone. Indeed, a usable 
split menu must be combined with a traditionally ordered menu \textit{and} must 
display the most frequently selected items as a \textit{hot list} at the top of 
this 
traditional menu.\newline

Split menu was widely adopted and implemented in many user interfaces. The most 
renowned split menu is the font menu displayed in Microsoft Word which 
was already used by \textsc{Sears} and \textsc{Shneiderman} to conduct their in 
situ usability studies. However we must wait the early 2000’s to find 
interesting studies about their concrete contextualized efficiency. Paula 
\textsc{Selvidge} \cite{paula} conducted an intensive test to assess split menu 
performance. It was based on 112 tasks to be performed by 73 individuals. She 
proved that split menu was achieving a faster completion time than static menu 
but she also pointed out that they both had the same error rate. Finally she 
identified a clear preference for split menu among participants. James R. 
\textsc{Warren} and Patrick \textsc{Bolton} \cite{warren_bolton} proved that 
split menu was more effective than static menu in the context of ophthalmologic 
diagnoses. They conducted a test with several doctors and found out that a small 
list of 20 diagnoses was mostly used. In this context, a split menu was indeed 
more efficient than a static menu. Mona \textsc{Tom} et al. \cite{mona} proved 
that split menu was also very efficient for automotive mobile multimedia 
applications. Mahamad \textsc{Saipunidzam} et al. \cite{saipu} embedded a split 
menu into a web browser address bar for data entry purposes. They proved that 
(1) it was easier to access Internet and find the desired address through split 
menu and (2) 80\% of participants agreed with the implementation of such a 
split menu in their web browser. In conclusion, many studies confirmed the 
importance and usefulness of split menu to improve both user experience and 
user performance.

\section{Adaptive vs. Adaptable approach}

During the 2000’s, two antagonist notions - respectively called 
\textit{adaptive} and \textit{adaptable} menus - were also prone to discussions 
and researches. Some findings are unclear and highly nuanced but they are still 
worth investigating.\newline

Khalid \textsc{Al-Omar} and Dimitrios \textsc{Rigas} \cite{alomar1} defined the 
concept of \textit{adaptive menu} as \enquote{a system-controlled menu that 
dynamically changes the interface layout and content to each user’s needs}. 
These menus tend to use \textit{graphical} or \textit{spatial techniques} to 
reduce visual search time. A \textit{spatial technique} consists of 
recognizing the most interesting items and copying/moving them for easier 
access. Split menus are based on this notion such that the 
most frequently selected items are moved to the top of the menu. A 
\textit{graphical technique} consists of recognizing the most interesting items 
and then modifying their graphical representation. For example, the most 
frequently selected items could be boldfaced to become catchy for the eyes. 
Both authors defined the concept of \textit{adaptable menu} as \enquote{a 
user-controlled menu that provides techniques which permit the users to adjust 
their layout and content to suit their needs}. Some adaptable menus are said to 
be \textit{coarse-grained} and allow users to move interesting items up or down 
the menu. Others are said to be \textit{fine-grained} and allow users to move 
directly these interesting items to a specific position in the menu. Khalid 
\textsc{Al-Omar} and Dimitrios \textsc{Rigas} finally insisted on a third notion 
called 
\textit{mixed-initiative} which has the ability to combine both adaptable and 
adaptive approaches at the same time. For example, by providing a menu that 
either displays the most frequent or the most recent items in the top section 
and such that this criteria is chosen by the user itself. This type of menu 
can be considered as adaptable and adaptive at the same time because it is 
customized by the system according to the user’s choice. Adaptable, 
adaptive and mixed-initiative menu organizations are depicted by Figure 
\ref{fig:adapt_organizations}.\newline

\begin{figure}[!ht]
    \centering
    \begin{tikzpicture}[node distance=0.8cm]
        \tikzstyle{item}
	  =[draw, fill=gray!20, rectangle, text width=2cm, 
	  minimum height=0.5cm, text centered]
	\tikzstyle{button}
	=[draw, square, text width=0.5cm, minimum height=0.6cm, text centered]
	\tikzstyle{radio_button}
	=[draw, square, text width=0.2cm, minimum height=0.4cm, text centered]
	
	\node[item] (A2)
	  {Item H};
	\node[item] (B2)
	  [below=3pt of A2] {Item A};
	\node[item] (C2)
	  [below=3pt of B2] {Item B};
	\node[item] (D2)
	  [below=3pt of C2] {Item C};
	\node[item] (E2)
	  [below=3pt of D2] {Item D};
	\node[item] (F2)
	  [below=3pt of E2] {Item E};
	\node[item] (G2)
	  [below=3pt of F2] {Item F};
	\node[item] (H2)
	  [below=3pt of G2] {Item G};
	\node[item] (J2)
	  [below=3pt of H2] {Item I};
	\node[item] (H2)
	  [below=3pt of J2] {Item J};
	\node (1)
	  [above=2.3cm of A2] {(1)};
	\node (user)
	  [above=1.8cm of A2] {User-controlled};
	\node (empty)
	  [above=18pt of A2] {};
	\node[button] (up)
	  [above left=3pt and -0.8cm of A2]{$\wedge$};
	\node[button] (down)
	  [above right=3pt and -0.8cm of A2]{$\vee$};
	
	\draw[-latex] (user)--(empty);
	
	\node[item,fill=gray!60] (A)
	  [right=2cm of A2]{Item E};
	\node[item,fill=gray!60] (B)
	  [below=3pt of A] {Item G};
	\node[item,fill=gray!60] (C)
	  [below=3pt of B] {Item I};
	\node[item,fill=gray!60] (D)
	  [below=3pt of C] {Item J};
	\node[item] (E)
	  [below=3pt of D] {Item A};
	\node[item] (F)
	  [below=3pt of E] {Item B};
	\node[item] (G)
	  [below=3pt of F] {Item C};
	\node[item] (H)
	  [below=3pt of G] {Item D};
	\node[item] (J)
	  [below=3pt of H] {Item F};
	\node[item] (H1)
	  [below=3pt of J] {Item H};
	\node (2)
	  [above=2.3cm of A] {(2)};
	\node (syst)
	  [above=1.8cm of A] {System-controlled};
	\node (empty)
	  [above=1pt of A] {};
	  
	\draw[-latex] (syst)--(empty);
	
	\node[item,fill=gray!60] (A3)
	  [right=2cm of A] {Item E};
	\node[item,fill=gray!60] (B3)
	  [below=3pt of A3] {Item G};
	\node[item,fill=gray!60] (C3)
	  [below=3pt of B3] {Item I};
	\node[item,fill=gray!60] (D3)
	  [below=3pt of C3] {Item J};
	\node[item] (E3)
	  [below=3pt of D3] {Item A};
	\node[item] (F3)
	  [below=3pt of E3] {Item B};
	\node[item] (G3)
	  [below=3pt of F3] {Item C};
	\node[item] (H3)
	  [below=3pt of G3] {Item D};
	\node[item] (J3)
	  [below=3pt of H3] {Item F};
	\node[item] (H3)
	  [below=3pt of J3] {Item H};
	\node (3)
	  [above=2.3cm of A3] {(3)};
	\node (user3)
	  [above=1.8cm of A3] {User-controlled};
	\node (empty3)
	  [above=25pt of A3] {};
	\node (empty4)
	  [right=1pt of A3] {};
	\node (syst3)
	  [below right=2pt and -4pt of user3] {System-controlled};
	\node[radio_button] (rec)
	  [above left=3pt and -0.5cm of A3] {};
	\node[radio_button] (freq)
	  [above=3pt of rec] {x};
	\node (freq_text)
	  [right=3pt of freq]{Frequency};
	\node (rec_text)
	  [right=3pt of rec]{Recency};
	  
	\draw[-latex] (user3)--(empty3);
	\draw[-latex] (syst3)|-(empty4);
	  
    \end{tikzpicture}
    \caption{(1) Adaptable, (2) adaptive and (3) mixed-initiative menu 
organizations with a focus on system-controlled and user-controlled features.}
    \label{fig:adapt_organizations}
\end{figure}

Many studies were performed before the work of Khalid \textsc{Al-Omar} and 
Dimitrios 
\textsc{Rigas}. A first controlled lab study compared user performance between a 
static, 
an adaptive and an adaptable split menu. The 
findings were the following: (1) 55\% of the 27 subjects preferred the 
adaptable menu over the adaptive and static ones, (2) the static and adaptable 
split menus were equally performant but allowed a faster selection time than 
the 
adaptive one. A second study involved the traditional Microsoft Word font menu, 
and an adaptable and an adaptive versions of this menu. The experiment proved 
that (1) 65\% of the subjects preferred the adaptable menu, (2) 20\% preferred 
the default font menu and (3) 15\% preferred the adaptive menu. However it is 
important to put these conclusions into perspective because (1) both adaptable 
and adaptive interfaces were very different from the traditional font menu and 
(2) all participants used the adaptable interface before the adaptive approach 
such that they were already used to their own customization choices. A third 
study proved that (1) the adaptable approach was the most performant one, (2) 
adaptive and static split menu were equally performant, (3) 55\% of subjects 
preferred the adaptable approach \textit{if and only if} they received guidance 
to use it efficiently, (4) 30\% of subjects preferred the adaptive alternative 
and (5) 15\% preferred the traditional split menu. In conclusion, the adaptable 
approach seems to be the most preferred and the most performant alternative. 
However the results vary a lot from one study to another in terms of adaptive 
and static menus. Guidance also seems to be an interesting prerequisite to 
enhance user performance and user experience.\newline

Another set of studies restrained their focus on adaptive and static split 
menus. The first study required to perform a set of telephone directory 
searches. The authors proved that the adaptive menu reduced the selection time 
by 35\% and the error rate by 40\% in comparison to the static menu. Moreover 
the adaptive menu was preferred by 69\% of the participants. A second study 
presented different results. Indeed, the traditional static approach provided 
better time performance than the adaptive alternative for the first group 
of tasks. It’s only during the second part of the experiment that both menus 
were considered equally performant. In this case, it is interesting to point 
out that users preferred the static approach to the adaptive one. In conclusion, 
it seems that the adaptive approach still needs a few tweaks in order to prove 
a concrete usefulness. Notice that both a period of adjustment and guidance 
seem essential for users to become performant with new types of menu 
organization.\newline

In 2005, Teophanis \textsc{Tsandilas} \cite{tsandilas} managed to explain why 
the 
adaptive approach was often disliked by users. He conducted an experiment based 
on 2 different adaptation techniques respectively called \textit{highlighting 
suggestions} and \textit{shrinking non-suggested items}. The first technique 
consisted of highlighting 4 to 8 suggested items based on the user’s cursor 
position. The second one consisted of highlighting 4 to 8 suggested items while 
shrinking the undesired items with a fisheye lens method. Both menu 
organizations are depicted by figure \ref{fig:tsandilas_menus} which 
highlights items A, B, F and H as suggested items. Teophanis \textsc{Tsandilas} 
proved 
that the effectiveness of adaptive techniques may vary along with the accuracy 
of their prediction mechanism. Indeed, a low prediction accuracy leads the 
users 
to untrust the system. And, of course, user performance vary a lot according to 
this level of accuracy. In conclusion, Teophanis \textsc{Tsandilas} explained 
that 
adaptive mechanisms had to be carefully implemented and tested to 
prove efficient enough.\newline

\begin{figure}[!ht]
    \centering
    \begin{tikzpicture}[node distance=0.8cm]
        \tikzstyle{item}
	  =[draw, fill=gray!20, rectangle, text width=2cm, 
	  minimum height=0.5cm, text centered]
	\tikzstyle{item-xs}
	  =[draw, fill=gray!20, rectangle, text width=2cm, 
	  minimum height=0.1cm, text centered]
	
	\node[] (1)
	  {(1)};
	\node[item,fill=gray!50] (A)
	  [below=3pt of 1]{Item A};
	\node[item,fill=gray!50] (B)
	  [below=3pt of A] {Item B};
	\node[item] (C)
	  [below=3pt of B] {Item C};
	\node[item] (D)
	  [below=3pt of C] {Item D};
	\node[item] (E)
	  [below=3pt of D] {Item E};
	\node[item,fill=gray!50] (F)
	  [below=3pt of E] {Item F};
	\node[item] (G)
	  [below=3pt of F] {Item G};
	\node[item,fill=gray!50] (H)
	  [below=3pt of G] {Item H};
	\node[item] (I)
	  [below=3pt of H] {Item I};
	\node[item] (J)
	  [below=3pt of I] {Item J};
	  
	\node[] (2)
	  [right=3cm of 1] {(2)};
	\node[item,fill=gray!50] (A2)
	  [below=3pt of 2] {Item A};
	\node[item,fill=gray!50] (B2)
	  [below=3pt of A2] {Item B};
	\node[item-xs] (C2)
	  [below=3pt of B2] {\tiny Item C};
	\node[item-xs] (D2)
	  [below=3pt of C2] {\tiny Item D};
	\node[item-xs] (E2)
	  [below=3pt of D2] {\tiny Item E};
	\node[item,fill=gray!50] (F2)
	  [below=3pt of E2] {Item F};
	\node[item-xs] (G2)
	  [below=3pt of F2] {\tiny Item G};
	\node[item,fill=gray!50] (H2)
	  [below=3pt of G2] {Item H};
	\node[item-xs] (I2)
	  [below=3pt of H2] {\tiny Item I};
	\node[item-xs] (J2)
	  [below=3pt of I2] {\tiny Item J};
    \end{tikzpicture}
    \caption{\textsc{Tsandilas}' menu organizations with 10 items and 
considering that 
items A, B, F and H are the suggested items : (1) highlighting suggestions and 
(2) shrinking non-suggested items.}
    \label{fig:tsandilas_menus}
\end{figure}

In 2010, Khalid \textsc{Al-Omar} and Dimitrios \textsc{Rigas} went even 
further. They conducted 
an experiment based on 5 distinct menu organizations: (1) adaptable menu, (2) 
adaptive split menu, (3) adaptive/adaptable highlighted menu, (3) 
adaptive/adaptable minimised menu and (4) mixed-initiative menu. These new menu 
organizations are depicted by Figure \ref{fig:khalid_menus}. It is interesting 
to notice that the adaptive split menu was divided into 2 top sections. They 
were respectively dedicated to frequency and recency ordering. Each 
organization was respectively tested with a large menu of 29 items and with a 
small menu of 17 items. Both authors found out that user satisfaction was 
affected by the size of the menu. In overall the adaptive/adaptable minimised 
menu was the most preferred approach for the small menu, followed by the 
adaptable and adaptive/adaptable highlighted menus. The worst alternatives were 
the mixed-initiative approach and the adaptive split menu. However the 
mixed-initiative approach was by far the most preferred one for the large menu, 
followed by the adaptive/adaptable minimised interface. Adaptive split menu, 
adaptable and adaptive/adaptable highlighted menus were mostly disliked by the 
participants for the large menu. Therefore Khalid \textsc{Al-Omar} and 
Dimitrios \textsc{Rigas} 
concluded that the \textit{size of personalised content} was a matter of 
concern for 
user satisfaction: (1) users preferred to have less control and receive more 
help from the system for large menus, (2) users preferred when desired items 
were moved to the top section and undesired ones were hidden from the menu and 
(3) the concept of split menu was less efficient for small menus. In other 
words, users preferred the adaptable interfaces for small menus because they 
are more likely to customize a small menu than a large one. Indeed, a large 
content requires more effort to be understood and used efficiently. Users also 
liked the idea of minimising the menu and reducing the number of displayed 
items. Finally, both authors made an interesting observation: the recency 
criteria was not used by the test participants. Using such a criteria implies to 
update the menu each time a selection is performed. Therefore, the menu is 
constantly being modified and it becomes very confusing for users.\newline

\begin{figure}[!ht]
    \centering
    \begin{tikzpicture}[node distance=0.8cm]
        \tikzstyle{item}
	  =[draw, fill=gray!20, rectangle, text width=2cm, 
	  minimum height=0.5cm, text centered]
	\tikzstyle{button}
	=[draw, square, text width=0.5cm, minimum height=0.6cm, text centered]
	\tikzstyle{radio_button}
	=[draw, square, text width=0.2cm, minimum height=0.4cm, text centered]
	
	\node[item] (A)
	  {Item B};
	\node[item] (B)
	  [below=3pt of A] {Item A};
	\node[item] (C)
	  [below=3pt of B] {Item C};
	\node[item] (D)
	  [below=3pt of C] {Item D};
	\node[item] (E)
	  [below=3pt of D] {Item J};
	\node[item] (F)
	  [below=3pt of E] {Item F};
	\node[item] (G)
	  [below=3pt of F] {Item G};
	\node[item] (H)
	  [below=3pt of G] {Item H};
	\node[item] (I)
	  [below=3pt of H] {Item I};
	\node[item] (J)
	  [below=3pt of I] {Item E};
	\node[button] (up)
	  [above left=3pt and -0.8cm of A]{$\wedge$};
	\node[button] (down)
	  [above right=3pt and -0.8cm of A]{$\vee$};  
	\node[] (1)
	  [above=1.2cm of A]{(1)};
	
	\node[item,fill=gray!60] (A2)
	  [right=0.5cm of A]{Item B};
	\node[item,fill=gray!60] (B2)
	  [below=3pt of A2] {Item F};
	\node[item,fill=gray!85] (C2)
	  [below=3pt of B2] {Item A};
	\node[item,fill=gray!85] (D2)
	  [below=3pt of C2] {Item H};
	\node[item] (E2)
	  [below=3pt of D2] {Item C};
	\node[item] (F2)
	  [below=3pt of E2] {Item D};
	\node[item] (G2)
	  [below=3pt of F2] {Item E};
	\node[item] (H2)
	  [below=3pt of G2] {Item G};
	\node[item] (I2)
	  [below=3pt of H2] {Item I};
	\node[item] (J2)
	  [below=3pt of I2] {Item J};
	\node[] (2)
	  [above=1.2cm of A2]{(2)};
	
	\node[item,fill=gray!60] (A3)
	  [right=0.5cm of A2]{Item B};
	\node[item] (B3)
	  [below=3pt of A3] {Item A};
	\node[item] (C3)
	  [below=3pt of B3] {Item C};
	\node[item,fill=gray!60] (D3)
	  [below=3pt of C3] {Item D};
	\node[item] (E3)
	  [below=3pt of D3] {Item J};
	\node[item,fill=gray!60] (F3)
	  [below=3pt of E3] {Item F};
	\node[item,fill=gray!60] (G3)
	  [below=3pt of F3] {Item G};
	\node[item] (H3)
	  [below=3pt of G3] {Item H};
	\node[item] (I3)
	  [below=3pt of H3] {Item I};
	\node[item] (J3)
	  [below=3pt of I3] {Item E};
	\node[button] (up)
	  [above left=3pt and -0.8cm of A3]{$\wedge$};
	\node[button] (down)
	  [above right=3pt and -0.8cm of A3]{$\vee$};  
	\node[] (3)
	  [above=1.2cm of A3]{(3)};
	  
	\node[item] (A4)
	  [right=0.5cm of A3]{Item B};
	\node[item] (B4)
	  [below=3pt of A4] {Item D};
	\node[item] (C4)
	  [below=3pt of B4] {Item F};
	\node[item] (D4)
	  [below=3pt of C4] {Item G};
	\node[item] (E4)
	  [below=3pt of D4] {Item A};
	\node[item] (F4)
	  [below=3pt of E4] {Item C};
	\node[button] (down4)
	  [below=3pt of F4] {$\vee$};
	\node[] (4)
	  [above=1.2cm of A4]{(4)};
	  
	\node[item,fill=gray!60] (A5)
	  [right=0.5cm of A4]{Item B};
	\node[item,fill=gray!60] (B5)
	  [below=3pt of A5] {Item D};
	\node[item,fill=gray!60] (C5)
	  [below=3pt of B5] {Item F};
	\node[item,fill=gray!60] (D5)
	  [below=3pt of C5] {Item G};
	\node[item] (E5)
	  [below=3pt of D5] {Item A};
	\node[item] (F5)
	  [below=3pt of E5] {Item C};
	\node[item] (G5)
	  [below=3pt of F5] {Item E};
	\node[item] (H5)
	  [below=3pt of G5] {Item H};
	\node[item] (I5)
	  [below=3pt of H5] {Item I};
	\node[item] (J5)
	  [below=3pt of I5] {Item J};
	\node[] (5)
	  [above=1.2cm of A5]{(5)};
	\node[radio_button] (rec)
	  [above left=3pt and -0.5cm of A5] {};
	\node[radio_button] (freq)
	  [above=3pt of rec] {x};
	\node (freq_text)
	  [right=3pt of freq]{Frequency};
	\node (rec_text)
	  [right=3pt of rec]{Recency};
    \end{tikzpicture}
    \caption{\textsc{Al-Omar} and \textsc{Rigas}' menu organizations with 10 
items : (1) 
adaptable, (2) adaptive split, (3) adaptive/adaptable highlighted, (4) 
adaptive/adaptable minimised and (5) mixed-initiative menus.}
    \label{fig:khalid_menus}
\end{figure}

\section{Responsive menus}
In 2009, Yusuke \textsc{Fukazawa} et al. \cite{fukazawa} focused their 
researches on 
menus for mobile phones, observing an increasing complexity in mobile 
interfaces. They conducted an experiment that last 6 days with 20 participants. 
Each subject was given an additional Windows mobile phone configured with 3 
distinct menu organizations described as \textit{responsive}. A 
\textit{responsive menu} is designed with the idea that mobile screens are 
smaller than desktop computers and that menus should therefore be adapted to 
this constraint. \textsc{Fukazawa} et al. developed 3 responsive menu 
organizations (see Figure \ref{fig:fukazawa_menus}). The first one was made of 
2 columns of items and was displaying the most important items at the top. The 
last two menus were displaying varying item sizes: the more important the item 
was, the bigger it was displayed. Yusuke \textsc{Fukazawa} et al. observed that 
70\% of the participants were both satisfied by the new menu organizations and 
by the prediction function used to identify important items. However, they 
noticed that \textit{master users} showed higher preferences for old menus 
and \textit{novice users} showed higher satisfaction for the new 
menu organizations.\newline

\begin{figure}[!ht]
    \centering
    \begin{tikzpicture}[node distance=0.8cm]
	\tikzstyle{item}
	  =[draw, fill=gray!20, rectangle, text width=2cm, 
	  minimum height=0.5cm, text centered]
	  
        \tikzstyle{item_col}
	  =[draw, fill=gray!20, rectangle, text width=1.3cm, 
	  minimum height=0.5cm, text centered]
	\tikzstyle{item_col_xl}
	  =[draw, fill=gray!20, rectangle, text width=1.3cm, 
	  minimum height=1cm, text centered]
	  
	\tikzstyle{item_circle}
	  =[draw, fill=gray!20, circle, text width=0.3cm, text centered]
	\tikzstyle{item_circle_xl}
	  =[draw, fill=gray!20, circle, text width=0.8cm, text centered]
	  
	\node[item_col, fill=gray!50] (A)
	  {Item C};
	\node[item_col] (B)
	  [below=3pt of A] {Item A};
	\node[item_col] (C)
	  [below=3pt of B] {Item D};
	\node[item_col] (D)
	  [below=3pt of C] {Item F};
	\node[item_col] (E)
	  [below=3pt of D] {Item I};
	\node[item_col, fill=gray!50] (F)
	  [right=3pt of A] {Item G};
	\node[item_col] (G)
	  [below=3pt of F] {Item B};
	\node[item_col] (H)
	  [below=3pt of G] {Item E};
	\node[item_col] (I)
	  [below=3pt of H] {Item H};
	\node[item_col] (J)
	  [below=3pt of I] {Item J};
	\node[] (1)
	  [above right=0.7cm and -9pt of A]{(1)};
	  
	\node[item_col] (A2)
	  [right=1cm of F]{Item A};
	\node[item_col_xl] (B2)
	  [right=0pt of A2] {Item B};
	\node[item_col] (C2)
	  [right=0pt of B2] {Item C};
	\node[item_col_xl] (D2)
	  [below=0pt of A2] {Item D};
	\node[item_col] (E2)
	  [right=0pt of D2] {Item E};
	\node[item_col] (F2)
	  [right=0pt of E2] {Item F};
	\node[item_col] (G2)
	  [below=0pt of D2] {Item G};
	\node[item_col_xl] (H2)
	  [right=0pt of G2] {Item H};
	\node[item_col_xl] (I2)
	  [right=0pt of H2] {Item I};
	\node[item_col] (J2)
	  [below=0pt of G2] {Item J};
	\node[] (2)
	  [above right=0.7cm and -9pt of A2]{(2)};
	  
	\node[item_circle] (A3)
	  [right=1cm of C2]{A};
	\node[item_circle_xl] (B3)
	  [right=0pt of A3] { B};
	\node[item_circle] (C3)
	  [right=0pt of B3] { C};
	\node[item_circle_xl] (D3)
	  [below=0pt of A3] { D};
	\node[item_circle] (E3)
	  [right=0pt of D3] { E};
	\node[item_circle] (F3)
	  [right=0pt of E3] { F};
	\node[item_circle] (G3)
	  [below=0pt of D3] { G};
	\node[item_circle_xl] (H3)
	  [right=0pt of G3] { H};
	\node[item_circle_xl] (I3)
	  [right=0pt of H3] { I};
	\node[item_circle] (J3)
	  [below=0pt of G3] { J};
	\node[] (3)
	  [above right=0.7cm and -9pt of A3]{(3)};
	  
	
    \end{tikzpicture}
    \caption{\textsc{Fukazawa}'s menu organizations with 10 items : (1) 
2-column, (2) rectangle-shaped items and (3) circle-shaped items.}
    \label{fig:fukazawa_menus}
\end{figure}

\section{Learning outcomes}
Lots of interesting findings have been highlighted by these previous 
researches. 
First, \textsc{Card} proved that a \textit{meaningful organization} of menu 
items such 
as alphabetical or categorical ordering was beneficial for user experience. 
\textsc{Somberg} later proved that \textit{positionally constant} menus were 
preferred 
by users such that items should not move too much within the menu. He also 
identified that a \textit{period of adjustment} was required to handle 
efficiently new menu organizations. Then, \textsc{Sears} and 
\textsc{Shneiderman} managed to find 
the right balance between dynamic and static menus by introducing the concept 
of 
\textit{split menu}. It is a great example that combines almost positionally 
constant menu, frequency-ordering and meaningful organization. Experiments 
conducted before Khalid \textsc{Al-Omar} and Dimitrios \textsc{Rigas} proved 
that the
\textit{adaptable approach} was the most preferred alternative. They also 
proved that both a \textit{period of adjustment} and \textit{guidance} were 
necessary prerequisites for users to understand new menu organizations. 
Teophanis \textsc{Tsandilas} later explained that the \textit{accuracy 
prediction} of 
the adaptive mechanism was critical for the adaptive approach to be performant. 
Finally, Khalid \textsc{Al-Omar} and Dimitrios \textsc{Rigas} proved that users 
showed preferences for their \textit{minimised menu} because it was hiding 
unwanted items and therefore displaying less items at the same time. They also 
proved that the split menu was less efficient for small menus and highlighted 
the fact that alternatives may be required for small screens. Yusuke 
\textsc{Fukazawa} et al. investigated this possibility and implemented 
\textit{responsive} menus. They identified varying reactions between 
\textit{master} and \textit{novice users}. 
Indeed, master users liked to have more control and novice users preferred to 
be 
helped by the system. Master users also showed preferences for traditional menus 
because they were already use to them.
