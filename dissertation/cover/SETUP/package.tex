% ------------------------------
% PACKAGES DE BASE
% ------------------------------
\usepackage[utf8x]{inputenc}
\usepackage[T1]{fontenc}

%\usepackage[pdftex]{graphicx}
\usepackage[explicit]{titlesec}
\usepackage{lmodern}
\usepackage{graphics}
\usepackage[top=2.5cm,bottom=2.5cm,left=2.5cm,right=2.5cm]{geometry}
%\usepackage[top=2.5cm, bottom=2.5cm, outer=1cm, inner=2.1cm]{geometry}
%\usepackage{float} %Pour les figures avec [H]
\usepackage[colorinlistoftodos]{todonotes}
%------------------------------
% MATHS
% ------------------------------
\RequirePackage{amsmath}
\RequirePackage{amssymb}
\RequirePackage{amsthm}
\RequirePackage{mathrsfs}
\usepackage{numprint}
\usepackage{gensymb}
\usepackage[french]{nomencl}
\usepackage{url}
\usepackage{eurosym}

\usepackage{tabularx}

% ------------------------------
% COLOR
% ------------------------------
\RequirePackage{color}
\definecolor{UCLblue}{cmyk}{1.00,0.68,0.00,0.54}
\definecolor{LSMGold}{RGB}{155,128,13}
\definecolor{EPLblue}{cmyk}{0.70,0.30,0.00,0.00}
\definecolor{mygreen}{RGB}{36,139,34} % color values Red, Green, Blue
\definecolor{mylilas}{RGB}{161,32,247}
\definecolor{light-gray}{gray}{0.9}

% ------------------------------
% PERSONNALISATION
% ------------------------------
\RequirePackage[colorlinks=false]{hyperref}
\RequirePackage{listings}
\RequirePackage{graphicx}
\RequirePackage{fancyhdr}
\RequirePackage{fancybox}
\usepackage{enumitem}
%\RequirePackage{subfigure}
\RequirePackage{caption}
\RequirePackage{subcaption}
\RequirePackage{epstopdf}
\RequirePackage{xcolor,colortbl}
\usepackage{array,multirow,makecell}
\RequirePackage{pstricks} % Pour faire des figures POST-SCRIPT
\RequirePackage{titlesec}
\RequirePackage{tikz}
\usepackage{courier} % font de \texttt{•} plus jolie \fontfamily{phv}
\usepackage{lipsum}
\setcounter{secnumdepth}{3}
\usepackage{footnote}
\makesavenoteenv{tabular}
\makesavenoteenv{table}

% ------------------------------
% LANGUAGES
% ------------------------------

\lstdefinestyle{matlab} {language=matlab,style=numbers,style=MyFrame,style=Classic}
\lstset{
  language=matlab, 
  morekeywords={matlab2tikz},
  commentstyle=\color{mygreen}, % comment style
  keywordstyle=\color{blue},    % keyword style
  rulecolor=\color{black},    % if not set, the frame-color may be changed on line-breaks
  rulesepcolor=\color{black},
  stringstyle=\color{mylilas} % string style
}

% ------------------------------
% INFO SUR LE DOCUMENT
% ------------------------------
\renewcommand\title{Conception, design and testing of a split menu\\ for 
smartphone}	% Titre du TFE
%\newcommand\subtitle{Sous-titre éventuel}

\newcommand\nameone{Nathan \textsc{Magrofuoco}}	% Nom de l'étudiant
\newcommand\speciality{Management [60 ects]}		% Spécialité 
%\newcommand\options{Option(s) si requise(s)}	% À mentioner si demandé par la commission de programme
\newcommand\supervisor{Jean \textsc{Vanderdonckt}}	% Nom du promoteur
%\newcommand\cosupervisor{Prénom \textsc{Nom}}	% Nom du co-promoteur éventuel
%\newcommand\readerone{??? \textsc{???}}		% Nom du 1er lecteur
%\newcommand\readertwo{??? \textsc{???}}		% Nom du 2nd lecteur
%\newcommand\readerthree{??? \textsc{???}}	% Nom du 3e lecteur éventuel
\newcommand\years{2016-2017}	% Année académique

% ------------------------------
% COMANDE PERSO
% ------------------------------
% Création de la page de titre
\makeatletter
\renewcommand{\maketitle}
	{\begin{titlepage}
	\newgeometry{top=1.25cm,bottom=1.25cm,left=1.25cm,right=1.25cm}
	\begin{center}
		\includegraphics[width=\textwidth,trim={0 1cm 0 
0},clip]{banner_lsm.png}
	\end{center}
	\vspace*{9pt}
	\begin{flushright}
	    \color{UCLblue} \fontfamily{phv} \selectfont
	    {\huge \title} \\
	    \vspace*{12pt}
	    %{\Large \subtitle} \\
	    \vspace*{12pt}
		\large Dissertation presented by \\
		\textbf{\nameone}
		\\
		\vspace*{12pt} 
		for obtaining the Master's degree in  \\
		\textbf{\speciality} \\
%		\textit{Option(s): \options}	% Uncomment if necessary
		\vspace*{12pt}
		Supervisor\\
		\textbf{\supervisor} 
%		\textbf{, \cosupervisor}		% Uncomment if necessary
		\\
%		\vspace*{12pt}
%		Lecteurs \\
%		\textbf{\readerone, \readertwo}
%		\textbf{, \readerthree}	% Uncomment if necessary
%		\\
		\vspace*{12pt}
		Academic year \years \\
	\end{flushright}
	\vspace*{9pt}
	\color{LSMGold}{\rule{18.5cm}{8.5cm}}
  \end{titlepage}}
\makeatother

% Création de la préface
\def\beforepreface{
	%\color{UCLblue}
	\pagenumbering{roman}
	\pagestyle{plain}}
\def\prefacesection#1{%
	%\chapter*{#1}
	%\addcontentsline{toc}{chapter}{#1}
	}
\def\afterpreface{
	\color{black}
	\tableofcontents
	\newpage
	\printnomenclature
	\addcontentsline{toc}{chapter}{Liste des symboles}
	\newpage
	\listoffigures
	\addcontentsline{toc}{chapter}{Table des figures}
	%\newpage
	%\listoftables
	%\addcontentsline{toc}{chapter}{Liste des tableaux}
	\newpage
	\cleardoublepage
	\pagenumbering{arabic}
	\pagestyle{headings}}

% Création Back cover page
\def\makebackcoverpage{
	\newpage 
	\thispagestyle{empty}                    
	\newgeometry{top=1.25cm,bottom=1.25cm,left=1.25cm,right=1.25cm}
	\vspace*{17.6cm}
	\noindent {\footnotesize \color{UCLblue}  \selectfont Place des Doyens, 1-1348 Louvain-la-Neuve ~ ~ \color{LSMGold} \textbf{www.uclouvain.be/lsm} \\
	\vspace*{6pt}
	\color{LSMGold}{\rule{18.5cm}{8.25cm}}}}

% Back cover page sur page paire
\newcommand*\cleartoleftpage{%
  \clearpage
  \ifodd\value{page}\hbox{}\vspace*{\fill}\thispagestyle{empty}\newpage\fi
}

	
% Création des chapitres
\newlength\chapnumbA
\setlength\chapnumbA{2cm} 
\newlength\chapnumbB
\setlength\chapnumbB{0cm}
\titleformat{\chapter}[block]
{\normalfont}{}{0pt}
{\parbox[b]{\chapnumbA}{%
   \fontsize{50}{10}\selectfont\thechapter}  %taille_chiffre
  \parbox[b]{\dimexpr\textwidth-\chapnumbA\relax}{%
    \raggedleft%
    \hfill{\Huge#1}\\
    \rule{\dimexpr\textwidth-\chapnumbA\relax}{0.4pt}}}
\titleformat{name=\chapter,numberless}[block]
{\normalfont}{}{0pt}
{\parbox[b]{\chapnumbB}{%
   \mbox{}}%
  \parbox[b]{\dimexpr\textwidth-\chapnumbB\relax}{%
    \raggedleft%
    \hfill{\Huge#1}\\
    \rule{\dimexpr\textwidth-\chapnumbB\relax}{0.4pt}}}
    
    
% Nomenclature : unité à droite
\newcommand{\nomunit}[1]{%
 \renewcommand{\nomentryend}{\hspace*{\fill}#1}}    
	
\linespread{1.25}