\chapter{Introduction}
Since the initial drawings of Charles \textsc{Babbage} and the first modern 
computer 
built by Konrad \textsc{Zuse}, computer technology has widely evolved and 
spread around 
the globe. Initially dedicated and used by a few IT professionals, 
the advent of personal computing radically changed the game and made everyone a 
potential computer user \cite{hci}. Computer providers faced serious usability 
issues with the initial command dialogs. Some still recognized brands took 
advantage of innovative and creative minds to overcome these problems.\newline

Unfortunately the effectiveness of their solutions has often not been 
scientifically proven. In the early 1980s, some computer scientists showed 
great 
interest into assessing the usability of the traditionally designed UI's. An 
area of research and practice called \enquote{\textit{human-computer 
interaction}} - and 
commonly referred to as \textit{HCI} - started to emerge along with these 
scientists. They focused on studying the quality and quantity of information 
transfer between humans and computers. Later they started to publish their own 
mathematical models and related UI solutions to enhance UX.\newline

Last decade saw the rise of mobile technologies. Minimising computer components 
has always been a common interest for computer manufacturers. Smartphone in the 
hand, tablet in the other. People have become keen of these microtechnologies. 
Unfortunately they raised new issues. Among them, the complexification of UI has 
become an increasing problem for users, especially for novices. We come to a 
point where humans themselves must adapt to the technology and HCI has still a 
long road ahead.

\section{Objectives}
This master thesis was initially focused on the conception, the design and the 
testing of a split menu for smartphone. Starting with the guidelines proposed 
by \textsc{Sears} and \textsc{Shneiderman} \cite{sears} almost 20 years ago, the 
first objective 
was to design a \textit{usable} split menu in the context of a small 
touching screen. Assessing \textit{usability} requires to be attentive to 3 
interesting properties : (1) effectiveness, (2) efficiency and (3) 
satisfaction. A controlled experiment was planned to be conducted in order to 
evaluate these properties and compare a traditional menu organization to a 
split menu designed for smartphone. \newline

The conception and the design of such a split menu also required to update the 
guidelines published by \textsc{Sears} and \textsc{Shneiderman}. Indeed these 
guidelines haven't been modified since 1994. An intense review of the field of 
study brang us to take into account diverse 
approaches. Many researchers and studies have therefore influenced the conducted 
experiment. The master thesis is finally more about conceiving, designing and 
testing new menu organizations for smartphone.

\section{Structure of the written dissertation}
The written dissertation is organized as follows:

\begin{enumerate}
  \item \textbf{State of the Art}: the first chapter is the starting point of 
the study. It provides a review of the field of research and describes the 
knowledge base over which the entire experiment was built.
  \item \textbf{Methodology}: the second chapter describes the methodology used 
to conceive the experiment. First, it presents the initial hypotheses. Then, it 
describes the experimental method. Finally it argues the Android app used 
during the study.
  \item \textbf{Results}: the third chapter provides an overview of the results 
gathered during the experiment, then it analyzes these results to 
confirm, reverse and update the initial hypotheses.
  \item \textbf{Conclusion}: finally, a conclusion ends the written 
dissertation by discussing the confirmed hypotheses, the encoutered issues, 
the unchallenged matters and the future improvements to be made.
\end{enumerate}


